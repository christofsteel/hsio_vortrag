\documentclass{beamer}
\usetheme{_tuhs/tuhs}
\usepackage[utf8x]{inputenc}
\usepackage[T1]{fontenc}
\usepackage{ucs}
\usepackage{amsmath}
\usepackage{amsfonts}
\usepackage{amssymb}
\usepackage{enumerate}
\usepackage{listings}
\usepackage{multicol}
\usepackage{xcolor}

\newcommand{\code}[2]
{
	\begin{block}{#1: #2}
	\lstinputlisting[language=#1]{#2}
	\end{block}
}

\newcommand{\outputcode}[3]
{
	\begin{block}{#1: #2}
	\lstinputlisting[language=#1]{#2}
	\begin{description}
	\item[Ausgabe]\texttt{#3}
	\end{description}
	\end{block}
}

\lstnewenvironment{bash}
{    
    \lstset{language=bash}   
}
{
}

\lstnewenvironment{haskell}
{
	\lstset{language=haskell}
}
{
}
%
%\newenvironment{haskell}[1]
%{
%    \begin{block}{Haskell: #1}
%    \begin{lstlisting}
%    \lstset{language=haskell}
%}
%{
%    \end{lstlisting}
%    \end{block}
%}

\title{Haskell in the real World: \\ The IO monad}
\author{Christoph ``Hammy`` Stahl}
\begin{document}
\begin{frame}
	\maketitle
\end{frame}
\begin{frame}{Überblick}
\begin{multicols}{2}
\begin{tiny}
\tableofcontents
\end{tiny}
\end{multicols}
\end{frame}

\section{Einleitung}
\subsection{Software}
\begin{frame}[<+->]{Software}
Software:
\begin{itemize}
\item Linux
\item GHC
\item cabal
\item Download: \textcolor{blue}{\url{http://hackage.haskell.org/platform/}}\\
\item Oder aus dem eigenen Paketmanager.
\end{itemize}
\end{frame}

\begin{frame}[<+->][fragile]{cabal}
\begin{itemize}
\item Cabal ist ein Paketmanager f\"ur Haskell
\item Benutzt, um Bibliotheken und Programme zu installieren, und dabei automatisch Abh\"angigkeiten aufzul\"osen,
\item Teilweise m\"ussen entsprechende C Bibliotheken installiert sein.
\item L\"ad die Programme und Abh\"angigkeiten aus dem \textcolor{blue}{\href{http://hackage.haskell.org}{Hackage}}.
\begin{block}{Shell:}
\begin{bash}
$ cabal install <paketname> 
\end{bash} %$
\end{block}
\end{itemize}
\end{frame}

\subsection{Hackage}
\begin{frame}[<+->]{Hackage}
\begin{itemize}
\item Sammlung von Haskell Bibliotheken und Programmen.
\item Online Zugriff auf die Dokumentation der Bibliotheken.
\item Vergleichbar mit CPAN oder RubyGems.
\end{itemize}
\end{frame}

\subsection{Was ist die IO Monade}
\begin{frame}[<+->]{Was ist die IO Monade}
\begin{itemize}
\item Kapselt IO Zugriff.
\item Kommunikation mit der "Realen" Welt.
\item M\"oglichkeit das Programm zu steuern.
\item Potential f\"ur Seiteneffekte.
\end{itemize}
\end{frame}

\subsection{main-Funktion}
\begin{frame}[<+->][fragile]{main-Funktion}
\begin{itemize}
\item Startpunkt eines jeden Haskell Programmes.
\begin{block}{Haskell:}
\begin{haskell}
main :: IO ()
\end{haskell}
\end{block}
\item Muss demnach ein \textbf{IO} () zurück geben.
\begin{block}{Beispiel:}
\begin{haskell}
main = do 
  foo <- barIOfunction "text"
  let stuff = map (pureFunction) foo
  putStrLn stuff -- putStrLn :: IO ()
\end{haskell}
\end{block}
\end{itemize}
\end{frame}

\section{stdin/stdout}
\begin{frame}[<+->][fragile]{stdin/stdout}
\begin{itemize}
\item Jedes Programm hat 3 standard Datenströme
\begin{small}
\begin{description}
\item[stdin] Standardeingabe, meist Tastatur.
\item[stdout] Standardausgabe, meist Terminal.
\item[stderr] Standardausgabe für Fehler, meist Terminal.
\end{description}
\end{small}
\item Können in Dateien oder an andere Programme gelenkt werden.
\begin{block}{Shell:}
\begin{bash}
$ prog1 < file | prog2 > file 2> error.log
\end{bash} %$
\end{block}
\end{itemize}
\end{frame}

\subsection{stdout}
\begin{frame}[<+->]{stdout}
Wichtige Funktionen:
\begin{description}
\item[putStr] Schreibt einen String in die Ausgabe. \\ \texttt{putStr :: String -> IO ()}
\item[putStrLn] Schreibt einen String in die Ausgabe, gefolgt von einem Newline. \\ \texttt{putStrLn :: String -> IO ()}
\item[putChar] Schreibt einen Buchstaben in die Ausgabe. \\ \texttt{putChar :: Char -> IO ()}
\item[print] Schreibt eine Variable in die Ausgabe, gefolgt von einem Newline. \\ \texttt{print :: Show a => a-> IO ()}
\end{description}
\end{frame}

\subsection{Beispiel: Hello World!}
\begin{frame}[<+->][fragile]{Beispiel: Hello World!}
\tiny
\begin{columns}
\column<0->{0.5\textwidth}
\outputcode{Haskell}{HelloWorld1.hs}{Hello \\ World!}
\outputcode{Haskell}{HelloWorld2.hs}{Hello World!}
\column<0->{0.5\textwidth}
\outputcode{Haskell}{Umstaendlich.hs}{Char!}
\outputcode{Haskell}{Add.hs}{7}
\end{columns}
\end{frame}

\subsection{stdin}
\begin{frame}[<+->]{stdin}
Wichtige Funktionen:
\begin{description}
\item[getLine] Liest eine Zeile von der Eingabe. \\ \texttt{getLine :: IO String}
\item[getChar] Liest ein Zeichen von der Eingabe. \\ \texttt{getChar :: IO Char}
\item[readLn] Liest eine Zeile von der Eingabe und versucht das Ergebnis zu interpretieren. \\ \texttt{readLn :: Read a => IO a}
\item Die Funktionen warten, bis Zeichen vorhanden sind.
\item[getContents] Liest alle Zeichen von der Eingabe. \textcolor{red}{(lazy)}\\ \texttt{getContents :: IO String}
\item Programmfluss geht weiter, bis auf die Eingabe zugegriffen wird.
\end{description}
\end{frame}


\subsection{Beispiel: Hello <User>}
\begin{frame}[<+->][fragile]{Beispiel: Hello <User>}
\outputcode{Haskell}{HelloUser.hs}{Name?\\ \textcolor{red}{Hammy}\\Hello Hammy}
\end{frame}

\section{Datenströme}
\subsection{Bibliothek: System.IO}
\begin{frame}{Bibliothek: System.IO}
\begin{itemize}
\item Standardbibliothek für IO in Haskell
\item Prelude exportiert nur einige Funktionen aus System.IO
\item Funktionen für Datei Zugriff
\item Allgemeiner: Auf Handles zugreifen
\end{itemize}
\end{frame}

\subsection{Handles}
\begin{frame}[<+->]{Handles}
Mit Handles kann man Datenströme verwalten
\begin{itemize}
\item Ein/Ausgabe
\item Dateien lesen/schreiben
\item In einen Netzwerkstream schreiben/aus einem Netzwerkstream lesen.
%\item Eigene Datenströme für dir Kommunikation im Multithreading.
\end{itemize}
\end{frame}

\subsection{Eigenschaften von Handles}
\begin{frame}[<+->]{Eigenschaften von Handles}
Ein Handle kann
\begin{itemize}
\item Eingabe(input), Ausgabe(output) oder Beides verwalten
\item offen(open), geschlossen(closed) oder halbgeschlossen(semi-closed) sein
\end{itemize}
%\pause
%stdin, stdout und stderr sind Handles.
\end{frame}

\subsection{Operationen auf Handles}
\begin{frame}[<+->]{Operationen auf Handles}
\begin{description}
\item[hGetChar] Liest ein Zeichen von dem Handle \\ \texttt{hGetChar :: Handle -> IO Char}
\item[hGetLine] Liest einen String von dem Handle (bis zum Newline) \\ \texttt{hGetLine :: Handle -> IO String}
\item[hGetContents] Liest den Kompletten Handle aus \\ \texttt{hGetContents :: Handle -> IO String}
\item[hPutChar] Schreibt ein Zeichen in einen Handle \\ \texttt{hPutChar :: Handle -> Char -> IO ()}
\item[hPutStr] Schreibt einen String in einen Handle \\ \texttt{hPutStr :: Handle -> String -> IO ()}
\item[hPutStrLn] Schreibt einen String und ein Newline in einen Handle\\ \texttt{hPutStrLn :: Handle -> String -> IO ()}
\item[hClose] Schließt einen Handle. \\ \texttt{hClose :: Handle -> IO ()}
\end{description}
\end{frame}

\subsection{Datenstr\"ome sind Datenstr\"ome}
\begin{frame}[<+->]{Datenstr\"ome sind Datenstr\"ome}
\center
Operationen klingen vertraut?\\
\pause
stdin, stdout und stderr sind Handles (Definiert in System.IO)\\
\end{frame}

\begin{frame}[fragile]{Auszug aus System.IO}
\tiny
\code{Haskell}{AuszugAusSystemIO.hs}
\end{frame}

\section{Datei Zugriff}
\subsection{Dateien \"offnen}
\begin{frame}[<+->]{Dateien \"offnen}
\begin{description}
\item[openFile] Öffnet eine Datei zum Lesen/Schreiben oder Erweitern \\ \texttt{\small openFile :: FilePath -> IOMode -> IO Handle}
\item[FilePath] \texttt{type FilePath = String}
\item[IOMode] \texttt{data IOMode = ReadMode | WriteMode | AppendMode | ReadWriteMode}
\item[withFileDo]TEst123
\end{description}
\end{frame}

\subsection{Dateien einlesen}
\begin{frame}{Dateien einlesen}
\begin{description}
\item[readFile] Liest eine Datei ein \\ \texttt{readFile :: FilePath -> IO String}
\end{description}
\begin{itemize}
\item Liest die Datei mit \texttt{hGetContents} ein.
\item Datei wird Lazy gelesen.
\item Bis die Datei vollständig gelesen worden ist, ist der Handle in einem semiclosed Zustand
\item Gibt Probleme, wenn man mehrfach in eine Datei schreiben und aus einer Datei Lesen möchte
\end{itemize}
\end{frame}

\subsection{Dateien schreiben}
\begin{frame}{Dateien schreiben}
\begin{description}
\item[writeFile] Schreibt einen String in einer Datei \\ \texttt{writeFile :: FilePath -> String -> IO ()}
\item Überschreibt den gesamten Inhalt der Datei
\end{description}
\end{frame}

\subsection{Beispiel: Verschl\"usseler}
\begin{frame}{Beispiel: Verschl\"usseler}
\tiny
\code{Haskell}{hscrypt.hs}
\end{frame}

\section{Netzwerk}
\subsection{Bibliothek: Network}
\begin{frame}{Bibliothek: Network}
\begin{itemize}
\item Hochleveliger Zugriff auf Netzwerkfunktionen
\item Bietet auch einen Lowleveligen Zugriff mit \texttt{Network.Socket}
\item Netwerkströme werden als Handles aufgefasst
\item \textcolor{red}{Windows only} Benötigt \texttt{withSocketsDo :: IO a -> IO a}
\end{itemize}
\end{frame}

\subsection{Datentypen}
\begin{frame}{Datentypen}
\begin{description}
\item[PortID] \begin{itemize}
\item Ein Servicename: \texttt{Service String} \\ \textit{Bsp.: Service "ftp"}
\item Eine Portnummer: \texttt{PortNumber PortNumber} \\ \textit{Bsp.: PortNumber 1337}
\item \textcolor{green}{Unix only} Ein Unix Socket: \texttt{UnixSocket String} \\ \textit{UnixSocket "/home/hammy/theSocket"}
\end{itemize}
\item[HostName]
\begin{itemize}
\item Ein Hostname \textit{Bsp.: "haskell.org"}
\item Eine IPv4 Adresse \textit{Bsp.: "78.46.100.180"}
\item Eine IPv6 Adresse \textit{Bsp.: "2a01:4f8:121:6::10"}
\end{itemize}
\end{description}
\end{frame}

\subsection{Server}
\begin{frame}{Server}
\begin{description}
\item[listenOn] Öffnet einen Socket auf einem Port, und lauscht \\ \texttt{listenOn :: PortID -> IO Socket}
\item[accept] Wartet auf Connections auf einem Socket, und gibt einen Handle für diese Verbindung \\ \texttt{accept :: }
\end{description}
\end{frame}

\subsection{Client}
\begin{frame}{Client}
\end{frame}

\subsection{Beispiel: Einfacher Datei Transfer}
\begin{frame}{Beispiel: Einfacher Datei Transfer}
\end{frame}

\section{Threading}
\subsection{Bibliothek: Control.Concurrent}
\begin{frame}{Bibliothek: Control.Concurrent}
\end{frame}

\subsection{Thread erstellen}
\begin{frame}{Thread erstellen}
\end{frame}

\subsection{Gemeinsame Variablen}
\begin{frame}{Gemeinsame Variablen}
\end{frame}

\begin{frame}{IORef}
\end{frame}

\begin{frame}{MVar}
\end{frame}

\subsection{Beispiel: Multithreaded Chat Server}
\begin{frame}{Beispiel: Multithreaded Chat Server}
\end{frame}

\section{Coole Sachen}
\begin{frame}{Coole Sachen}
\begin{enumerate}
\item getArgs
\item cmdArgs
\item ByteString
\item Foreign Function Interface
\end{enumerate}
\end{frame}

\end{document}
