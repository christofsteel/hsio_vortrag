\documentclass{beamer}
\usetheme{_tuhs/tuhs}
\usepackage[utf8x]{inputenc}
\usepackage{ucs}
\usepackage{amsmath}
\usepackage{amsfonts}
\usepackage{amssymb}
\usepackage{enumerate}
\usepackage{listings}
\lstloadlanguages{Haskell}
\lstnewenvironment{code}
    {\lstset{}%
      \csname lst@SetFirstLabel\endcsname}
    {\csname lst@SaveFirstLabel\endcsname}
    \lstset{
      basicstyle=\small\ttfamily,
      flexiblecolumns=false,
      basewidth={0.5em,0.45em},
      literate={+}{{$+$}}1 {/}{{$/$}}1 {*}{{$*$}}1 {=}{{$=$}}1
               {>}{{$>$}}1 {<}{{$<$}}1 {\\}{{$\lambda$}}1
               {\\\\}{{\char`\\\char`\\}}1
               {->}{{$\rightarrow$}}2 {>=}{{$\geq$}}2 {<-}{{$\leftarrow$}}2
               {<=}{{$\leq$}}2 {=>}{{$\Rightarrow$}}2 
               {\ .}{{$\circ$}}2 {\ .\ }{{$\circ$}}2
               {>>}{{>>}}2 {>>=}{{>>=}}2
               {|}{{$\mid$}}1               
    }


\title{Haskell in the real World: \\ The IO monad}
\author{Christoph ``Hammy`` Stahl}
\begin{document}
\begin{frame}
	\maketitle
\end{frame}
\begin{frame}{Überblick}
	\tableofcontents
\end{frame}

\section{Einleitung}
\subsection{ghc / cabal install}
\begin{frame}{ghc / cabal install}

\end{frame}

\subsection{hackage}
\begin{frame}{hackage}[fragile]
\begin{code}
main :: IO ()
main = putStrLn "Hey there"
\end{code}
\end{frame}

\subsection{main}
\begin{frame}{main}
\end{frame}

\section{stdin/stdout}
\subsection{stdout}
\begin{frame}{stdout}
\end{frame}

\subsection{stdin}
\begin{frame}{stdin}
\end{frame}

\subsection{Beispiel: Hello World}
\begin{frame}{Beispiel: Hello World}
\end{frame}

\subsection{Beispiel: Hello <User>}
\begin{frame}{Beispiel: Hello <User>}
\end{frame}

\section{Datei Zugriff}
\subsection{Bibliothek: System.IO}
\begin{frame}{Bibliothek: System.IO}
\end{frame}

\subsection{Dateien öffnen}
\begin{frame}{Dateien öffnen}
\end{frame}

\subsection{Dateien einlesen}
\begin{frame}{Dateien einlesen}
\end{frame}

\subsection{Dateien schreiben}
\begin{frame}{Dateien schreiben}
\end{frame}

\subsection{Beispiel: Verschlüsseler}
\begin{frame}{Beispiel: Verschlüsseler}
\end{frame}

\section{Netzwerk}
\subsection{Bibliothek: Network}
\begin{frame}{Bibliothek: Network}
\end{frame}

\subsection{Server}
\begin{frame}{Server}
\end{frame}

\subsection{Client}
\begin{frame}{Client}
\end{frame}

\subsection{Beispiel: Einfacher Datei Transfer}
\begin{frame}{Beispiel: Einfacher Datei Transfer}
\end{frame}

\section{Threading}
\subsection{Bibliothek: Control.Concurrent}
\begin{frame}{Bibliothek: Control.Concurrent}
\end{frame}

\subsection{Thread erstellen}
\begin{frame}{Thread erstellen}
\end{frame}

\subsection{Gemeinsame Variablen}
\begin{frame}{Gemeinsame Variablen}
\end{frame}

\subsubsection{IORef}
\begin{frame}{IORef}
\end{frame}

\subsubsection{MVar}
\begin{frame}{MVar}
\end{frame}

\subsection{Beispiel: Multithreaded Chat Server}
\begin{frame}{Beispiel: Multithreaded Chat Server}
\end{frame}

\section{Coole Sachen}
\begin{frame}{Coole Sachen}
\begin{enumerate}
\item getArgs
\item cmdArgs
\item ByteString
\item Foreign Function Interface
\end{enumerate}
\end{frame}

\end{document}