\documentclass{beamer}
\usetheme{_tuhs/tuhs}
\usepackage[utf8x]{inputenc}
\usepackage[T1]{fontenc}
\usepackage{ucs}
\usepackage{amsmath}
\usepackage{amsfonts}
\usepackage{amssymb}
\usepackage{enumerate}
\usepackage{listings}
\usepackage{multicol}
\usepackage{xcolor}


\lstnewenvironment{bash}
{    
    \lstset{language=bash}   
}
{
}

\lstnewenvironment{haskell}
{
	\lstset{language=haskell}
}
{
}
%
%\newenvironment{haskell}[1]
%{
%    \begin{block}{Haskell: #1}
%    \begin{lstlisting}
%    \lstset{language=haskell}
%}
%{
%    \end{lstlisting}
%    \end{block}
%}

\title{Haskell in the real World: \\ The IO monad}
\author{Christoph ``Hammy`` Stahl}
\begin{document}
\begin{frame}
	\maketitle
\end{frame}
\begin{frame}{Überblick}
\begin{multicols}{2}
\begin{tiny}
\tableofcontents
\end{tiny}
\end{multicols}
\end{frame}

\section{Einleitung}
\subsection{Software}
\begin{frame}[<+->]{Software}
Software:
\begin{itemize}
\item Linux
\item GHC
\item cabal
\item Download: \url{http://hackage.haskell.org/platform/}\\
\item Oder aus dem eigenen Paketmanager.
\end{itemize}
\end{frame}

\begin{frame}[<+->][fragile]{cabal}
\begin{itemize}
\item Cabal ist ein Paketmanager f\"ur Haskell
\item Benutzt, um Bibliotheken und Programme zu installieren, und dabei automatisch Abh\"angigkeiten aufzul\"osen
\item Teilweise m\"ussen entsprechende C Bibliotheken installiert sein.
\item L\"ad die Programme und Abh\"angigkeiten aus dem \textcolor{blue}{\href{http://hackage.haskell.org}{Hackage}}
\begin{block}{Shell:}
\begin{bash}
$ cabal install <paketname> 
\end{bash} %$
\end{block}
\end{itemize}
\end{frame}

\subsection{Hackage}
\begin{frame}{Hackage}
\begin{itemize}
\item Sammlung von Haskell Bibliotheken und Programmen
\item Online Zugriff auf die Dokumentation der Bibliotheken
\item Vergleichbar mit CPAN oder RubyGems
\end{itemize}
\end{frame}

\subsection{Was ist die IO Monade}
\begin{frame}[<+->]{Was ist die IO Monade}
\begin{itemize}
\item Kapselt IO Zugriff.
\item Kommunikation mit der "Realen" Welt
\item M\"oglichkeit das Programm zu steuern
\item Potential f\"ur Kaputt /*TODO*/
\end{itemize}
\end{frame}

\subsection{main-Funktion}
\begin{frame}[<+->][fragile]{main-Funktion}
\begin{itemize}
\item Startpunkt eines jeden Haskell Programmes
\begin{block}{Haskell:}
\begin{haskell}
main :: IO ()
\end{haskell}
\end{block}
\item Muss demnach ein \textbf{IO} () zurück geben
\begin{block}{Beispiel:}
\begin{haskell}
main = do 
  foo <- barIOfunction "text"
  let stuff = map (pureFunction) foo
  putStrLn stuff -- putStrLn :: IO ()
\end{haskell}
\end{block}
\end{itemize}
\end{frame}

\section{stdin/stdout}
\subsection{stdout}
\begin{frame}[<+->][fragile]{stdout}
\begin{itemize}
\item 3 Standard Datenströme
\begin{small}
\begin{description}
\item[stdin] Standardeingabe, meist Tastatur
\item[stdout] Standardausgabe, meist Terminal
\item[stderr] Standardausgabe für Fehler, meist Terminal
\end{description}
\end{small}
\item Könnens in Dateien oder an andere Programme gelenkt werden.
\begin{block}{Shell:}
\begin{bash}
$ prog1 < file | prog2 > file
\end{bash} %$
\end{block}
\end{itemize}
\end{frame}

\subsection{stdin}
\begin{frame}{stdin}
\end{frame}

\subsection{Beispiel: Hello World}
\begin{frame}{Beispiel: Hello World}
\end{frame}

\subsection{Beispiel: Hello <User>}
\begin{frame}{Beispiel: Hello <User>}
\end{frame}

\section{Datei Zugriff}
\subsection{Bibliothek: System.IO}
\begin{frame}{Bibliothek: System.IO}
\end{frame}

\subsection{Dateien \"offnen}
\begin{frame}{Dateien \"offnen}
\end{frame}

\subsection{Dateien einlesen}
\begin{frame}{Dateien einlesen}
\end{frame}

\subsection{Dateien schreiben}
\begin{frame}{Dateien schreiben}
\end{frame}

\subsection{Beispiel: Verschl\"usseler}
\begin{frame}{Beispiel: Verschl\"usseler}
\end{frame}

\section{Netzwerk}
\subsection{Bibliothek: Network}
\begin{frame}{Bibliothek: Network}
\end{frame}

\subsection{Server}
\begin{frame}{Server}
\end{frame}

\subsection{Client}
\begin{frame}{Client}
\end{frame}

\subsection{Beispiel: Einfacher Datei Transfer}
\begin{frame}{Beispiel: Einfacher Datei Transfer}
\end{frame}

\section{Threading}
\subsection{Bibliothek: Control.Concurrent}
\begin{frame}{Bibliothek: Control.Concurrent}
\end{frame}

\subsection{Thread erstellen}
\begin{frame}{Thread erstellen}
\end{frame}

\subsection{Gemeinsame Variablen}
\begin{frame}{Gemeinsame Variablen}
\end{frame}

\begin{frame}{IORef}
\end{frame}

\begin{frame}{MVar}
\end{frame}

\subsection{Beispiel: Multithreaded Chat Server}
\begin{frame}{Beispiel: Multithreaded Chat Server}
\end{frame}

\section{Coole Sachen}
\begin{frame}{Coole Sachen}
\begin{enumerate}
\item getArgs
\item cmdArgs
\item ByteString
\item Foreign Function Interface
\end{enumerate}
\end{frame}

\end{document}
